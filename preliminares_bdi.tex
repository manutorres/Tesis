\subsection{Modelo BDI}

El \textit{modelo Creencia-Deseo-Intensión}, en adelante \textit{BDI} 
(\textit{Belief-Desire-Intention}),
es un modelo desarrollado para el diseño de agentes inteligentes, basado en una
vista simplificada de la inteligencia humana. Como se analizará en la sección 
\ref{sec:arquitecturaBDI}, el sistema presentado en este trabajo implementa una 
adaptación de dicho modelo. Por esta razón, se introducen en esta sección los 
conceptos básicos relacionados, que sirvieron de base para nuestro desarrollo.

El modelo BDI está dedicado al modelado formal del razonamiento práctico, es 
decir, la formalización de las bases y explicaciones psicológicas y filosóficas 
(provenientes principalmente de la filosofía de la mente y de la acción) de los 
conceptos de agente, acción, intención, creencia, voluntad, deliberación, 
razonamiento de medios y fines, etc. El razonamiento práctico es incorporado 
en agentes (por ejemplo, los seres humanos) capaces de perseguir y, por lo tanto, 
comprometerse con una determinada meta factible (una acción en particular) 
a través de una cuidadosa planificación de los medios, de las condiciones previas 
y las acciones que conducen a ese objetivo. 

Estos conceptos son incorporados al modelo mediante la implementación de los 
aspectos principales de la teoría del razonamiento práctico humano de Michael Bratman %REF
(también referido como Belief-Desire-Intention, o BDI). Es decir, implementa 
las nociones de creencia, deseo y (en particular) intensión,de una manera inspirada 
por Bratman. Una discusión más extensa puede ser encontrada en el mencionado 
trabajo de Bratman y en Searle (cf. [14]). %REF

Este basamento teórico permite al modelo resolver un problema particular que 
se presenta en la programación de agentes. Provee un mecanismo para separar la 
actividad de seleccionar un plan de la ejecución de los planes actualmente activos.
Los agentes BDI son capaces de balancear el tiempo invertido en deliberar sobre los
planes (elegir qué hacer) y ejecutar estos planes (llevarlo a cabo). La actividad 
de crear los planes en primera instancia, escapa al alcance del modelo.

\subsubsection{Creencias, Deseos e Intensiones}

Las \textit{creencias, deseos e intenciones} son consideradas estados mentales 
intencionales (de forma opuesta a, por ejemplo, el dolor o el placer). Las \textit{creencias} 
describen la percepción de la realidad a través de datos provenientes de 
los sentidos. Representan el estado \textit{informacional} del agente; comprenden
el conocimiento (tanto de sentido común como teórico) sobre el mundo, ya sea 
externo o interno. Están sujetas a revisión, lo que implica que pueden 
cambiar en el futuro, pueden ser rechazadas o agregadas. 

Los \textit{deseos} e \textit{intenciones}, pueden ser vistos como conceptos que 
se asemejan, aunque con algunas sutiles diferencias. Los deseos representan el 
estado \textit{motivacional} del agente; consisten en su voluntad de alcanzar 
ciertos objetivos o situaciones. Entre los deseos, se distingue la noción de 
\textit{meta}. Una meta es un deseo que ha sido adoptado por el agente para 
ser perseguido activamente. Esta definición impone la restricción de que el 
conjunto de metas, o deseos activos, debe ser consistente. 

Por último, el concepto de intención representa el estado \textit{deliberativo}
del agente, lo que el agente ha elegido hacer. Constituyen deseos para los cuales
el agente se ha comprometido. Es una noción más ligado al compromiso que es 
asumido, en función alcanzar los estados o situaciones deseadas. 

\subsubsection{Deliberación y planificación}

Por \textit{deliberación} entendemos lo que la literatura denomina \textit{silogismo práctico},
es decir, la inferencia de una intensión a partir de un conjunto de creencias y
deseos. Esto es, la selección de un deseo factible. Una \textit{desición} 
consiste en el último paso de este proceso de inferencia mediante el cual 
resulta electo uno de muchos deseos y potenciales intensiones. Es, por esto, un
concepto ligado directamente al de intensión. Definir una intensión implica, en 
términos de agentes implementados, comenzar la ejecución de un \textit{plan}.

Una \textit{acción} puede ser definida, intuitivamente, como la ejecución de
una operación que causa un determinado efecto o consecuencia sobre el entorno
en el cual se está desempeñando el agente. La \textit{planificación} consiste 
en la disposición de una secuencia de acciones con el fin de lograr una (o más)
de sus intensiones de alcanzar una meta. Los planes pueden ser complejos en 
mayor o menor medida, en función a la cantidad de acciones que contiene. En 
particular, los planes pueden contener otros planes, dado que satisfacer una
meta puede requerir la satisfacción de metas intermedias. Esto refleja que en 
el modelo de Bratman, inicialmente los planes son concebidos sólo parcialmente, 
y los detalles son incorporados a medida que progresa su ejecución.

%Practical Reasono

%Agents
