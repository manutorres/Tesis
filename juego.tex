\section{Descripción del Juego}

El \textit{Multi-Agent Programming Contest} es un concurso de programación de Inteligencia Artificial iniciado en el año 2005, organizado por la Clausthal University of Technology 
\footnote{Más información \texttt{www.tu-clausthal.de}}, con el objetivo de estimular la investigación en el área de desarrollo y programación de Sistemas Multi-Agente. Para ello, la competencia propone diferentes escenarios de juego de manera anual, que obligan a los participantes tanto a identificar y resolver problemas clave, como a explorar lenguajes, plataformas y herramientas de programación para Sistemas Multi-Agente.

\subsection{Escenario}

El escenario del año 2011 está formado por el mapa de un planeta representado mediante un grafo. 
Cada vértice del grafo es una locación válida (y tiene un valor determinado), y existen arcos (con diferente costo de energía) que permiten a un agente desplazarse de una locación a otra.

En cada ronda de la competición participan dos equipos rivales. Cada equipo posee un conjunto de agentes con diferentes roles preestablecidos (\textit{Explorador}, \textit{Saboteador}, \textit{Reparador}, \textit{Sentinela} e \textit{Inspector}). El rol de cada agente define tanto el conjunto de acciones que puede realizar, como sus características físicas (\textit{Energía}, \textit{Salud}, \textit{Fuerza} y \textit{Rango de Visión}).


[Screenshot de una simulación]

\subsection{Puntaje}

La simulación del juego se desarrolla por turnos, y en cada turno se otorga a los equipos una determinada cantidad de puntos según el estado de la simulación. El objetivo del juego es obtener la mayor cantidad de puntos posibles cuando la simulación termina.

Para obtener puntos, los agentes de cada uno de los equipos deben lograr formar \textit{``zonas''} en el mapa logrando posicionarse en diferentes locaciones de manera estratégica. La predominancia de un equipo sobre el otro en los nodos es determinada por un algoritmo bien definido para la competencia, y el valor de todos los nodos dominados por un equipo es el principal factor del puntaje otorgado en cada uno de los turnos de la simulación. Algunas otras situaciones, como el logro de determinados \textit{achievements}, pueden otorgar puntos adicionales al equipo.

\subsection{Acciones}

Todos los agentes tienen acciones en común que pueden realizar en cada uno de los turnos de la simulación:

\begin{itemize}
	\item \texttt{goto(X)}: el agente se desplaza hacia el nodo X, siempre y cuando exista un arco que conecte el nodo actual del agente con X, y dicho arco tenga un costo menor a la energía actual del agente.
	\item \texttt{survey(X)}: el agente recibe en su próxima percepción los costos de todos los arcos conectados al nodo en el que se encuentra actualmente.
	\item \texttt{buy(X)}: el agente utiliza el dinero obtenido a partir de los \textit{achievements} para aumentar el valor máximo de cualquiera de sus características físicas (Energía, Salud, Fuerza o Rango de visión) en 1 punto.
	\item \texttt{recharge}: el agente recupera el 20\% de su energía.
	\item \texttt{skip}: el agente pasa al turno siguiente sin realizar ningún tipo de acción.
\end{itemize}

Además, según el rol de cada agente, existen algunas acciones específicas que pueden realizar:

\begin{itemize}
	\item \texttt{attack(X)}: acción disponible únicamente para los Saboteadores; el agente ataca a un enemigo X, si dicho enemigo se encuentra en el mismo nodo. El ataque, de tener éxito, decrementa la energía del agente enemigo, pudiendo deshabilitarlo en caso de que ésta llegue a 0.
	\item \texttt{parry}: acción disponible únicamente para los Reparadores, Saboteadores y Sentinelas. La acción protege al agente de los ataques enemigos, impidiendo que éstos tengan éxito.
	\item \texttt{probe}: acción disponible únicamente para los Exploradores. El agente recibe en su próxima percepción el valor del nodo en el que se encuentra actualmente. Ésta acción no sólo resulta importante por conocer el valor del nodo, sino que además permite que, cuando el nodo es conquistado por el equipo, dicho valor se sume al total de puntos de la zona. Un nodo en el que no se realizó \textit{probe} suma únicamente 1 punto al valor total de la zona.
	\item \texttt{inspect}: acción disponible únicamente para los Inspectores. El inspector recibe en su próxima percepción la información física (Salud, Energía, Fuerza, Rango de visión) de todos los agentes enemigos que se encuentren en el mismo nodo que él, o en cualquier vecino directo.
	\item \texttt{repair(X)}: acción disponible únicamente para los Reparadores. El reparador aumenta el valor de la \emph{Salud} actual de su compañero de equipo X (volviendo a habilitarlo, en caso de que su Salud fuera 0).
\end{itemize}

