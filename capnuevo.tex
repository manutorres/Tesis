
\chapter{Relación con DeLPAP}

Este capítulo se mostrará DeLPAP, el cual es un lenguaje APL (Agent Programming 
Language) basado en DeLP, y siguiendo el espíritu de 3APL. Este lenguaje fue
desarrollado por el Dr. Sebastián Gottifredi en su tesis doctoral, por lo que 
aportó sus conocimientos y descubrimientos en el área, en el diseño del sistema
multi-agente. Este aporte se ve reflejado en las similitudes entre los dos 
sistemas, las cuales se mostrarán también en el transcurso del capítulo.

\section{DeLPAP}

DeLPAP es un lenguaje de programación de agentes declarativo que utiliza 
DeLP-Servers para ejecutar los programas que modelan a los agentes. Un 
DeLP-Server es una implementación \textit{standalone} de DeLP, la cual 
tiene una arquitectura de cliente-servidor. 

\subsection{DeLP-Servers}
Un DeLP-Server mantiene un programa DeLP, al cual los clientes le realizan consultas. 
Para responderlas, el DeLP-Server utilizará la información pública almacenada 
en él, en conjunto con información individual enviada por los clientes a 
través de la consulta, creando un contexto especial para ésta. Este contexto 
es conocimiento que el servidor utilizará solamente para esa consulta en particular, y no afectará consultas futuras.

\begin{definicion}[Consulta contextual]
Sea \PP\ = \SD, una consulta contextual para \PP\ es un par 
\cquery{(\addset,\remset)}{Q}, donde $Q$ es una consulta \DLP, \addset\ y 
\remset\ son un conjunto de literales no contradictorios.
\end{definicion}

Este tipo especial de consulta contextual agrega y remueve temporalmente
elementos de \PP. Los literales de \addset\ serán agregados como hechos a 
\SSet, siempre y cuando se pueda asegurar que se mantendrá la consistencia.
Los literales de \remset\ determinarán dos acciones: las reglas y hechos a ser 
removidos de \SSet, y las reglas rebatibles a ser removidas de \DD.

\subsection{Definición de DeLPAP}

En esta sección se mostrará como la argumentación rebatible puede ser 
integrada en APL utilizando DeLP-Servers. Se definirá un lenguaje de 
programación de agentes declarativo llamado DeLPAP. Este lenguaje está 
basado parcialmente en 3APL %citar
, donde las interacciones entre los componentes mentales son realizadas a
través de consultas (en el sentido de consulta a un programa lógico).
Siguiendo este modelo, se utilizarán consultas contextuales para modelar la 
interacción entre los componentes de los agentes.

\subsubsection{Sintaxis}

Los agentes en DeLPAP están caracterizados por tres componentes: creencias,
metas y un conjunto de esquemas de planes. % TODO: ver si esta traducción es correcta
Éstos representan los componentes del estado
mental del agente, y serán especificados por programas \DLP, las cuales
serán usadas para computar las creencias y las metas en cada ciclo 
deliberativo. Los esquemas de planes serán usados para generar planes que 
logren alcanzar las metas, utilizando acciones para interactuar con el 
entorno y modificar su estado interno.

\paragraph{Creencias}

Éstas son utilizadas para representar la información que el agente tiene
e infiere del mundo. Será denotada por un programa \DLP\ $\BB = (\SB, \DB)$.
Por esto, el diseñador de los agentes puede especificar información 
potencialmente contradictoria utilizando reglas rebatibles. %ej necesario?

La información que un agente recibe a través de la percepción será 
considerada también como una creencia, por lo que también podrá aparecer en el
cuerpo de una regla rebatible en la base de creencias. % ej??

Para referirse a una creencia desde uno de los otros componentes, se utiliza
un literal especial $B(L)$, siendo $L$\ un literal, llamado \textit{creecias
actuales}. $B(L)$\ denota que el agente cree en $L$.

\paragraph{Metas}

Son utilizadas para representar situaciones en el mundo que el agente quiere
que se realicen. La \textbf{base de metas} tiene conocimiento sobre las metas
del agente, y es utilizada para determinar cual de ellas es factible para él 
en cada ciclo deliberativo.

En programación de agentes, las metas pueden ser condicionales o 
incondicionales. Las últimas, denominadas también independientes, siempre son
adoptadas, a diferencia de las condicionales, que dependen de otras metas y 
creencias. Pueden también ser conflictivas, o sea, algunas metas no pueden
ser seleccionadas en el mismo momento. 

Como las creencias, las metas serán representadas con un programa \DLP, donde
las metas incondicionales serán representadas como reglas estrictas, y las 
condicionales como reglas rebatibles. 
%acá estoy sacando lo de utilizar negación fuerte 

\begin{definicion}{Base de metas]
Una base de metas es un programa \DLP\ $\GB = (\SG, \DG, \CCG)$, donde $\DG$
tiene reglas de la forma(\drule{L_h}{\bel{L_0}, \ldots, \bel{L_k}, L_{k+1}, 
\ldots, L_n})$_{k+n \geq 0}$\ y $L_i$\ es un literal.
\end{definicion}

Para referirse a las metas desde los otros componentes, vamos a usar literales
especiales $G(L)$ llamados \textit{current goals}. $G(L)$ denota que el agente
desea lograr $L$.

Basic actions

mental actions

criterion change actions

external actions

Plan

Plan rule