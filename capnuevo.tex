
\chapter{Relación con DeLPAP}

\label{sec:delpap}

Este capítulo se mostrará DeLPAP \cite{Gottifredi:2010:QAA:1948131.1948169}, 
el cual es un lenguaje APL (Agent Programming 
Language) basado en DeLP, y siguiendo el espíritu de 3APL,
así como su relación con el sistema descripto en esta tesis. Este lenguaje fue
desarrollado por el Dr. Sebastián Gottifredi en su tesis doctoral, por lo que 
aportó sus conocimientos y descubrimientos en el área, en el diseño del sistema
multi-agente. Este aporte se ve reflejado en las similitudes entre los dos 
sistemas, las cuales se mostrarán también en el transcurso del capítulo.

\section{DeLPAP}

DeLPAP es un lenguaje de programación de agentes declarativo que utiliza 
DeLP-Servers para ejecutar los programas que modelan a los agentes. Un 
DeLP-Server es una implementación \textit{standalone} de DeLP, la cual 
tiene una arquitectura de cliente-servidor. 

\subsection{DeLP-Servers}
Un DeLP-Server mantiene un programa DeLP, al cual los clientes le realizan consultas. 
Para responderlas, el DeLP-Server utilizará la información pública almacenada 
en él, en conjunto con información individual enviada por los clientes a 
través de la consulta, creando un contexto especial para ésta. Este contexto 
es conocimiento que el servidor utilizará solamente para esa consulta en particular, y no afectará consultas futuras.

\begin{definicion}[Consulta contextual]
Sea \PP\ = \SD, una consulta contextual para \PP\ es un par 
\cquery{(\addset,\remset)}{Q}, donde $Q$ es una consulta \DLP, \addset\ y 
\remset\ son un conjunto de literales no contradictorios.
\end{definicion}

Este tipo especial de consulta contextual agrega y remueve temporalmente
elementos de \PP. Los literales de \addset\ serán agregados como hechos a 
\SSet, siempre y cuando se pueda asegurar que se mantendrá la consistencia.
Los literales de \remset\ determinarán dos acciones: las reglas y hechos a ser 
removidos de \SSet, y las reglas rebatibles a ser removidas de \DD.

\subsection{Definición de DeLPAP}

En esta sección se mostrará como la argumentación rebatible puede ser 
integrada en APL utilizando DeLP-Servers. Se definirá un lenguaje de 
programación de agentes declarativo llamado DeLPAP. Este lenguaje está 
basado parcialmente en 3APL %citar
, donde las interacciones entre los componentes mentales son realizadas a
través de consultas (en el sentido de consulta a un programa lógico).
Siguiendo este modelo, se utilizarán consultas contextuales para modelar la 
interacción entre los componentes de los agentes.

\subsubsection{Sintaxis}

Los agentes en DeLPAP están caracterizados por tres componentes: creencias,
metas y un conjunto de esquemas de planes. % TODO: ver si esta traducción es correcta
Éstos representan los componentes del estado
mental del agente, y serán especificados por programas \DLP, las cuales
serán usadas para computar las creencias y las metas en cada ciclo 
deliberativo. Los esquemas de planes serán usados para generar planes que 
logren alcanzar las metas, utilizando acciones para interactuar con el 
entorno y modificar su estado interno.

\paragraph{Creencias}

Éstas son utilizadas para representar la información que el agente tiene
e infiere del mundo. Será denotada por un programa \DLP\ $\BB = (\SB, \DB)$.
Por esto, el diseñador de los agentes puede especificar información 
potencialmente contradictoria utilizando reglas rebatibles. %ej necesario?

La información que un agente recibe a través de la percepción será 
considerada también como una creencia, por lo que también podrá aparecer en el
cuerpo de una regla rebatible en la base de creencias. % ej??

Para referirse a una creencia desde uno de los otros componentes, se utiliza
un literal especial $B(L)$, siendo $L$\ un literal, llamado \textit{creecias
actuales}. $B(L)$\ denota que el agente cree en $L$.

\paragraph{Metas}

Son utilizadas para representar situaciones en el mundo que el agente quiere
que se realicen. La \textbf{base de metas} tiene conocimiento sobre las metas
del agente, y es utilizada para determinar cual de ellas es factible para él 
en cada ciclo deliberativo.

En programación de agentes, las metas pueden ser condicionales o 
incondicionales. Las últimas, denominadas también independientes, siempre son
adoptadas, a diferencia de las condicionales, que dependen de otras metas y 
creencias. Pueden también ser conflictivas, o sea, algunas metas no pueden
ser seleccionadas en el mismo momento. 

Como las creencias, las metas serán representadas con un programa \DLP, donde
las metas incondicionales serán representadas como reglas estrictas, y las 
condicionales como reglas rebatibles. 
%acá estoy sacando lo de utilizar negación fuerte 
%
%\begin{definicion}{Base de metas]
%Una base de metas es un programa \DLP\ $\GB = (\SG, \DG, \CCG)$, donde $\DG$
%tiene reglas de la forma(\drule{L_h}{\bel{L_0}, \ldots, \bel{L_k}, L_{k+1}, 
%\ldots, L_n})$_{k+n \geq 0}$\ y $L_i$\ es un literal.
%\end{definicion}

Para referirse a las metas desde los otros componentes, vamos a usar literales
especiales $G(L)$ llamadas \textit{metas actuales}. $G(L)$ denota que el agente
desea lograr $L$.

\paragraph{Acciones básicas}

Especifican las capacidades del agente, o sea, acciones que un agente puede realizar como parte de un
plan para lograr una situación deseable. Se presentarán tres tipos de acciones básicas: las acciones
mentales, las acciones de cambio de criterio y las acciones externas.

\subparagraph{Acciones mentales}

Actualiza la base de creencias o la base de metas cuando es ejecutado. Este tipo de acciones puede ser
usado para guardar infomación recibida desde otros agentes o entornos, para remover metas no deseadas,
cambiar el comportamiento de los agentes, o para almacenar datos temporarios.

\subparagraph{Acciones de cambio de criterio}

Cambian completamente los criterios de comparación de la base de creencias y la base de metas.

\subparagraph{Acciones externas}

Son utilizadas para interactuar con el entorno o para comunicarse. Se asume que los efectos de las acciones externas se determinarán por el entorno y pueden no llegar a ser conocidas por el agente antes.

\paragraph{Planes}

Son utilizadas para lograr metas. Se trata simplemente de una secuencia de acciones, de la forma 
$[a_1,\ldots,a_m]$. 

Para determinar cómo actuar, los agentes DeLPAP utilizarán
\textbf{reglas de planes}, que estan basadas en las \textit{reasoning rules} utilizadas por 3APL.
Estos resultados establecerán un mapeo entre metas y planes. Básicamente, una regla de plan
determinará el plan a ejectuar con el objetivo de lograr su meta asociada, dado que se hayan
satisfecho determinadas precondiciones.

\subsubsection{Semántica}

Se definirá la semántica de DeLPAP, con lo cual se introducen los procesos utilizados para manejar
el razonamiento práctico y la dinámica de ejecución de los agentes.

Una \textbf{configuración} de un agente DeLPAP mostrará una captura de los componentes del estado 
mental en un determinado momento. Sólo contienen información sobre los componentes que pueden ser 
modificados durante el tiempo de vida del agente (creencias, metas, planes y percepciones).

A partir de la configuración se pueden obtener las creencias actuales del agente en un determinado 
momento, los cuales serán los literales instanciados garantizados. En el proceso de garantizado de las
creencias actuales se deben tener en cuenta las percepciones. Por esto, se utilizarán consultas 
contextuales para agregar la información perceptual a la base de creencias.

La semántica de metas describe las metas que un agente quiere alcanzar en una dada configuración. 
Son denominadas \textbf{metas actuales}, y seran literales garantizados de la base de metas. Sin 
embargo, no es suficiente utilizar solamente la información de la base de metas; los argumentos para
las metas pueden involucrar creencias, como tambien metas previamente alcanzadas no deberian ser 
consideradas como metas actuales. Por esto, las creencias actuales son necesarias en el proceso de 
garantía de las metas. Nuevamente, se utiliza el concepto de consultas contextuales para modelar este
comportamiento.

Utilizando las creencias y metas actuales, el agente podra determinar cuál regla de planes es 
aplicable. Fundamentalmente, una regla de planes será aplicable si y sólo si sus metas asociadas son 
metas actuales y sus creencias asociadas son creencias actuales. Una vez que el plan aplicable es 
encontrado, el plan y la meta de la regla son adoptados. Sin embargo, las reglas de planes sólo serán 
aplicadas si no existe un plan adoptado. Cuando un plan es adoptado, el agente se compromete a 
alcanzar la meta y dicho plan. Se mantendra en ejecución hasta que se finalice o cuando la meta 
adoptada no sea más una meta actual.

\section{Comparación de \texttt{d3lp0r} con DeLPAP} %TODO: cambiar

La intentención de este capítulo es realizar una comparación entre los sistemas DeLPOP y DeLPAP. En la
sección anterior, se introdujeron los conceptos que son necesarios para comprender el diseño de 
DeLPAP, algunas definiciones (aunque sin entrar en detalles estrictamente formales), asi como la 
sintaxis y la semántica de DeLPAP. Se pueden observar varias similitudes entre los sistemas (en los 
cuales se ahondarán en el resto del capítulo), lo cual se debe al aporte del Dr. Gottifreddi, que guío
el diseño y desarrollo del sistema multi-agente.

La mayor diferencia es estructural. Mientras todo en DeLPAP se mantiene en un conjunto de programas 
\DLP\ , en \texttt{d3lp0r} se utilizan una serie de módulos que se comunican entre sí, pasándose información
(percepciones, creencias, resultados de cómputos, etc.). En DeLPAP también existe comunicación, en la 
forma de consultas contextuales. Esto no fue necesario en el módulo de toma de decisiones de \texttt{d3lp0r} 
(escrito en \DLP) ya que el programa encargado de realizar las consultas también se encarga 
previamente de cargar toda la información relevante en forma de reglas \DLP\ en dicho módulo.

\subsection{Creencias}

Dado que tanto nuestro sistema como el sistema en el cual DeLPAP está basado (3APL) están 
fundamentados dentro de los conceptos de la arquitectura BDI, como fue explicado en la sección
\ref{sec:preliminaresBDI}, se comparten los cimientos teóricos, los cuales serian las creencias, los 
deseos (en DeLPAP las metas) y las intenciones (en DeLPAP las metas actuales).

Sin embargo, la manera en la cual se implementan los sistemas es completamente distinto. \texttt{d3lp0r} 
mantiene módulos separados para el procesamiento de las creencias y la toma de decisiones, pero 
utilizando distintos lenguajes (Prolog y \DLP\ respectivamente).

La información de la percepción nunca llega de manera directa al módulo de \DLP, sino que previamente 
es procesada, en principio, por el cliente de Python, y luego por el módulo de generación de 
creencias. En este proceso, algunas de las creencias son equivalentes a la percepción, mientras otras
necesitan un procesamiento más avanzado (por ej. el algoritmo de coloreo o la búsqueda de caminos).
Esto no está soportado directamente por DeLPAP, donde se asume que toda creencia podra ser generada
por la base de creencias. En la práctica, esto puede ser un problema si este tipo de creencias son 
complicadas para ser resueltas en el paradigma lógico. (La generación de creencias está implementada 
en Prolog, pero no puede ser considerado dentro del paradigma lógico porque se utilizan operaciones
extra-lógicas como el \texttt{assert} y \texttt{retract}.)

Una característica compartida es la utilización de un functor $B(L)$, la cual es utilizada en ambos
sistemas para diferenciar las creencias del resto de las reglas.

\subsection{Metas}

El concepto de metas es similar al de deseos. La manera de consultar cuáles son los deseos con 
argumentos garantizados en \texttt{d3lp0r} es una consulta individual para cada uno, instanciado para cada
una de las posibles instanciaciones dada la situación actual del agente.

El concepto de metas condicionales e incondicionales no existe explícitamente en \texttt{d3lp0r}. Sin embargo,
el chequeo de las condiciones de corte en parte tiene que ver con un comportamiento similar. Entre
las condiciones se encuentran chequeos que se realizan siempre, independientemente de la intención 
actual; por ejemplo, que el agente sea atacado o se encuentre amenazado por un enemigo. Por otro lado,
dentro del módulo de toma de decisiones, existen deseos que, de manera compulsiva, tienen un peso
mayor del que cualquier otro deseo pueda llegar a lograr. 

A diferencia de las creencias, la característica de utilizar functores no se comparte en las metas de
DeLPAP y los deseos de \texttt{d3lp0r}. Esto se debe a que los deseos nunca necesitarán diferenciarse del resto
de los argumentos, ya que las consultas se intenciones se realizan en serie, consultando por cada
deseo posible.

\subsection{Acciones}

Solamente existe un paralelo directo entre las acciones externas de DeLPAP, y las acciones en \texttt{d3lp0r}, 
las cuales son prácticamente iguales. Cambios de los estados mentales se realizan en \texttt{d3lp0r}, a través
del módulo de creencias, con la particularidad que se recalculan todos los turnos (al finalizar el
turno, los agentes borran prácticamente las creencias de manera íntegra). Cambios en el los criterios
de comparación no se realizan nunca en \texttt{d3lp0r}, por lo que no existe paralelo con las acciones de
cambio de criterio de DeLPAP.

\subsection{Planes}

Claramente se ve el paralelo entre los planes de los dos sistemas. En ambos, se trata de listas de 
literales que representan acciones. las cuales se ejecutarán secuencialmente siempre y cuando el plan
se deba seguir ejecutando.

Con respecto a las reglas de planes de DeLPAP, en donde existen reglas que determinan cuáles son los
tipos de planes posibles para una meta particular, existe relación con el módulo de planificación de
\texttt{d3lp0r}, en el hecho que también cuenta con una serie de tipos de planes posibles, los cuales se 
seleccionan a partir del deseo seleccionado (la intención). La diferencia se encuentra en que los 
planes no se calcular íntegramente en la etapa de planificación, dado que los planes son generados en 
la etapa de generación de creencias, ya que la información sobre los planes asociados a cada deseo es
utilizada a la hora de decidir qué deseo sera seleccionado, en el módulo de toma de decisiones. Los 
planes quedan almacenados como creencias para luego ser seleccionados, simplificando a nivel trivial 
la complejidad de la etapa de planificación, pero complicando la etapa de generación de creencias. 
Esto se diseñó de esta manera para tener información certera a la hora de seleccionar un deseo, ya que
no existió posibilidad alguna de utilizar algún tipo de heurística, dado que el juego no aportaba
ninguna información extra, fuera del grafo del mapa.

La intención y el plan actual es almacenado en \texttt{d3lp0r} como una de las pocas creencias que se mantiene
entre turnos. En el caso que la intención deje de ser ejecutada, su plan se descarta. Esto es similar
en DeLPAP, ya que estos dos elementos se encuentran íntimamente ligados.

\subsection{Cuestiones generales}
%TODO: cambiar

La configuración de un agente en DeLPAP, como fue explicado anteriormente, mantiene el estado mental
del agente en un determinado momento en el tiempo. En \texttt{d3lp0r} estos momentos son los turnos, y se 
puede pensar que su estado mental se encuentra representado por todo el conjunto de creencias, deseos,
planes e intenciones que se generan a lo largo de un ciclo deliberativo. No es presentado de una 
manera tan formal, ya que, en principio, no fue necesario para la realización del sistema. En general,
gran parte de las diferencias entre los sistemas se deben al hecho que DeLPAP es un formalismo, y 
\texttt{d3lp0r} un sistema multi-agente implementado para resolver un problema en particular, con sus 
determinadas restricciones de tiempo de cómputo, entre otras.

Las cuestiones relativas a la comunicación entre los diferentes programas \DLP\ de DeLPAP, realizados
a través de consultas contextuales de los DeLP-Servers, fueron resueltas en general de una manera más
directa en \texttt{d3lp0r}, ya que todos los módulos pertenecen al mismo proceso; no existe una estructura
de cliente-servidor. Lo más cercano es el servidor de percepciones, que sólo se preocupa por lograr
consistencia entre las creencias compartidas entre los agentes. Esto es necesario, ya que cada agente
corre en un proceso separado (potencialmente en una computadora separada, ya que la comunicación se
realiza a través de la red). 

En el diseño de \texttt{d3lp0r} se tuvo en cuenta también el rendimiento del sistema, con énfasis en la 
velocidad. Los turnos son cortos (dos segundos), en el cual se tiene que tener en cuenta el ida y 
vuelta entre el servidor y el agente. Este factor fue primordial en la competencia misma, ya que el 
servidor del juego se encontraba en Alemania, por lo que el retardo rondaba los $300 ms$. Se trató de
evitar utilizar otros retardos internos innecesarios. Comunicación entre los agentes debía haber, y 
éstos debían ser independientes, ya que era la intención tanto de la competencia como nuestra, por lo
que el servidor de percepciones era necesario, a pesar que agregara \textit{overhead}, tanto de 
comunicación como de sincronización (el servidor debia recibir todas las percepciones de todos los 
agentes antes de enviarlas nuevamente). 

Esta cuestión nos hizo alejarnos de la utilización del DeLP-Server, el cual es la manera principal
que se usa para ejecutar programas \DLP. Gracias a la colaboración de los miembros del LIDIA, pudimos 
utilizar el código fuente del intérprete de \DLP, el cual está implementado en Prolog, por lo que su
utilización fue directa. Gracias a este diseño se pudo evitar tener overhead de más en la utilización
de \DLP, lo cual terminó siendo uno de los componentes más eficientes del sistema (los cuellos de 
botella fueron otros, principalmente en el módulo de generación de creencias, como por ejemplo el 
algoritmo de coloreo, o el algoritmo de búsqueda de caminos).