\subsection{Programación Lógica Rebatible}

%\subsubsection{Representación de conocimiento}

A continuación introducimos las definiciones básicas necesarias para representar conocimiento
en Programación Lógica Rebatible (\DLP). Para un tratamiento exhaustivo, se remite al lector
interesado a %REF
. En lo que sigue, se asume que el lector posee un conocimiento básico acerca de los aspectos
fundamentales de la programación lógica.

\begin{definicion}(Programa \DLP\ \PP)
	Un programa lógico rebatible (delp) es un conjunto \PP\ = \SD\ donde \SSet\ y \DD\ representan conjuntos
	de conocimiento \textit{estricto} y \textit{rebatible}, respectivamente. El conjunto \SSet\ de 
	conocimiento estricto involucra \textit{reglas estrictas} de la forma \srule{L}{Q_1,\ldots,Q_k} y 					\textit{hechos} (reglas estrictas con cuerpo vacío), y se asume que es \textit{no-contradictorio}. 
	El conjunto \DD\ de conocimiento rebatible involucra \textit{reglas rebatibles} de la forma 
	\drule{L}{Q_1,\ldots,Q_k}, lo cual se interpreta como ``$Q_1,\ldots,Q_k$ proveen razones tentativas 
	para creer $L$''. Las reglas estrictas y rebatibles en \DLP\ son definidas usando un conjunto 
	finito de literales. Un literal es un átomo ($L$), la negación estricta de un átomo ($\sim L$) o 
	la negación \textit{default} de un átomo (\textit{not} $L$).
\end{definicion}

El lenguaje lógico subyacente en \DLP\ es el de la programación lógica extendida, %REF
enriquecido con el símbolo especial ``\drule{}{}'' para denotar reglas rebatibles. Tanto la negación 
\textit{default} como la clásica están permitidas (denotadas \textit{not} y \textit{$\sim$}, respectivamente).
Sintácticamente, el símbolo ``\drule{}{}'' es lo único que distingue un regla \textit{rebatible} 
\drule{L}{Q_1,\ldots,Q_k} de una regla \textit{estricta} (no-rebatible) \srule{L}{Q_1,\ldots,Q_k}. 
Las reglas \DLP\, por lo tanto, son consideradas como \textit{reglas de inferencia} en lugar implicaciones.
De forma análoga a la programación lógica tradicional, la \textit{definición} de un predicado $P$ en \PP ,
denotado $P^{\scriptsize{\PP}}$, está dada por el conjunto de todas las reglas (estrictas y rebatibles) con cabeza $P$ 
y aridad $n$ en \PP . Si $P$ es un predicado en \PP , entonces \textit{nombre(P)} y \textit{aridad(P)} denotan
el nombre y la aridad del predicado, respectivamente. Escribiremos \textsf{Pred}(\PP) para denotar el conjunto
de todos los nombres de predicados definidos en un programa \DLP\ \PP.

\subsubsection{Argumento, Contraargumento y Derrota}

Dado un programa \DLP\ \PP\ = \SD\, resolver consultas resulta en la construcción de \textit{argumentos}.
Un argumento \ArgA\ es un conjunto (posiblemente vacío) de reglas rebatibles fijas que junto al conjunto \SSet
provee una prueba lógica para un dado literal \ArgQ, satisfaciendo los requerimientos adicionales de 
\textit{no-contradicción} y \textit{minimalidad}. Formalmente:

\begin{definicion}[Argumento]
	Dado un programa \DLP\ \PP, un argumento \ArgA\ para una consulta \ArgQ, notado \AQ\, es un subconjunto de 
	instancias fijas de las reglas rebatibles en \PP, tal que:
	
	\begin{enumerate}[(1)]
		\item existe una derivación rebatible para \ArgQ de \SyA;
		\item \SyA\ es no-contradictorio (\ie, \SyA\ no implica dos literales complementarios $L$ y \lit{\no L}
		(o $L$ y \textsf{not}\ $L$), y,		
		\item \ArgA\ es minimal con respecto al conjunto inclusión (\ie, no hay \Ap\ $\subset$ \ArgA\ tal que
		existe una derivación rebatible para \ArgQ\ de \SyAp).
	\end{enumerate}
	
\end{definicion}

Un argumento \AaQa\ es un \textit{sub-argumento} de otro argumento \AbQb\ si $\ArgAa \subseteq \ArgAb$.
Dado un programa \DLP\ \PP, \textit{Args(\PP)} denota el conjunto de todos los posibles argumentos que 
pueden ser derivados de \PP.

La noción de derivación rebatible corresponde a la usual derivación SLD dirigida por consultas
empleada en programación lógica, aplicando \textit{backward chaining} a las reglas estrictas y rebatibles;
en este contexto, un literal negado \lit{\no P} es tratado simplemente como un nuevo nombre de predicado \textit{no\_P}. La minimalidad impone una especie de ``principio de la navaja de Occam'' sobre la construcción 
de argumentos. El requerimiento de no-contradicción prohíbe el uso de (instancias fijas de) reglas rebatibles
en un argumento \ArgA\ cuando \SyA\ deriva dos literales complementarios. Es de notar que el concepto de no-contradicción captura los dos enfoques usuales de negación en la programación lógica (negación \textit{default}
y negación clásica), ambas presentes en \DLP\ y relacionadas a la noción de contraargumento, como se muestra a continuación.

\begin{definicion}[\textbf{Contraargumento}]
	Un argumento \AaQa\ es un \textit{contraargumento} para un argumento \AbQb\ si y sólo si
	
	\begin{enumerate}[a)]
	\item (ataque a subargumento) existe un subargumento \AQ\ de \AbQb\ (llamado 
	\textit{subargumento en desacuerdo}) tal que el conjunto \SyQaQ\ es contradictorio, o
	\item (ataque por negación default) un literal \negda{\ArgQa}\ está presente en el cuerpo de alguna 
	regla en \ArgAb.
	\end{enumerate}	
	
\end{definicion}

%La primer noción de ataque es tomada del framework de Simari-Loui; la última está relacionada al 
%enfoque argumentativo de programación lógica de Dung, asi como también a otras formalizaciones,
%como el trabajo de Prakken y Sartor, o el trabajo de Kowa y Toni.

Como en muchos marcos de argumentación, vamos a asumir un \textit{criterio de preferencia} para los 
argumentos en conflicto definido como la relación $\preceq$, la cual es un subconjunto del producto 
cartesiano \textit{Args(\PP)} $\times$ \textit{Args(\PP)}. Esto lleva a la noción de \textit{derrota} entre argumentos como una refinación del criterio de contraargumento. En particular, vamos a distinguir entre 
dos tipos de derrotadores, \textit{propios} y \textit{por bloqueo}.

\begin{definicion}[\textbf{Derrotadores propios y por bloqueo]}]
	Un argumento \AaQa\ es un \textit{derrotador propio} para un argumento \AbQb\ si \AaQa\ contra-argumenta
	\AbQb\ con un sub-argumento en desacuerdo \AQ\ (ataque a subargumento) y \AaQa\ es estrictamente 
	preferido sobre \AQ\ con respecto a $\preceq$.
	
	Un argumento \AaQa\ es un \textit{derrotador por bloqueo} para un argumento \AbQb\ si \AaQa\ contra-argumenta
	\AbQb\ y una de las siguientes situaciones se presenta: (a) Hay un sub-argumento en desacuerdo \AQ\ para
	\AbQb, y \AaQa\ y \AQ\ no están relacionados entre sí con respecto a $\preceq$; o (b) \AaQa\ es un ataque 
	por negación default sobre algún literal \negda{\ArgQa}\ en \AbQb.
	
	El término \textit{derrotador} será usado para referirse indistintamente a derrotadores propios o por bloqueo.
\end{definicion}

La especificidad generalizada es típicamente usada como un criterio de preferencia basado en la sintaxis 
para argumentos en conflicto, favoreciendo aquellos argumentos que están \textit{más informados} o son 
\textit{más directos}. A modo de ejemplo, consideremos tres argumentos 

\nlA{\AS{\drule{a}{b,c}}{a}}, \AS{\drule{\no $a$}{b}}{\no a} y \AS{(\drule{a}{b});(\drule{b}{c})}{a}
construidos sobre la base de un programa \PP\ = \SD\ = ($\{b,c\},\{\drule{b}{c};\drule{a}{b};\drule{a}{b,c};\drule{\no a}{b}\}$).
Si se utiliza especificidad generalizada como criterio de comparación entre argumentos, el argumento 
\AS{\drule{a}{b,c}}{a} sería preferido sobre el argumento \AS{\drule{\no $a$}{b}}{\no a} ya que el primero es considerado \textit{más informado} (\ie, está basado en más premisas). Sin embargo, el argumento 
\AS{\drule{\no $a$}{b}}{\no a} es preferido sobre \AS{(\drule{a}{b});(\drule{b}{c})}{a} ya que el primero es considerado \textit{más directo} (\ie, es obtenido a partir de una derivación más corta). Sin embargo, debe
ser remarcado que, además de especificidad, otros criterios de preferencia alternativos pueden ser usados; 
\eg, aplicar prioridad sobre las reglas para definir la comparación de argumentos, o considerar valores 
numéricos correspondientes a medidas asociadas a conclusiones de argumentos. El primer enfoque es empleado en
{\footnotesize D}-P{\footnotesize ROLOG}, Lógica Rebatible, extensiones de la Lógica Rebatible, y programación lógica sin negación por falla.
El segundo criterio fue el aplicado en el desarrollo del sistema que se presenta en los capítulos siguientes.

\subsubsection{Computo de garantías a través de análisis dialéctico}

Dado un argumento \AQ, pueden existir diferentes derrotadores $\BaQa\ldots\BkQk$, k $\ge$ 0 para
\AQ. Si el argumento \AQ\ es derrotado, entonces ya no estaría soportando su conclusión \ArgQ. 
Sin embargo, dado que los derrotadores son argumentos, estos pueden a su vez ser derrotados. Esto 
induce un análisis dialéctico recursivo completo para determinar que argumentos son derrotados en 
última instancia. Para caracterizar este proceso, primero se introducen algunas nociones auxiliares.

Una \textit{línea argumentativa} comenzando en un argumento \AoQo\ (denotado $\lambda^{\scriptsize \AoQo}$)
es una secuencia [\AoQo, \AaQa, \AbQb,\ldots,\AnQn\ldots] que puede ser pensada como un intercambio de 
argumentos entre dos partes, un \textit{proponente} (argumentos en posiciones pares) y un \textit{oponente} (argumentos en posiciones impares). Cada \AiQi\ es un derrotador para el argumento previo \AimQim\ en la secuencia, $i > 0$.
A fin de evitar razonamiento \textit{falaz} o mal-formado (\eg , lineas argumentativas infinitas), el 
análisis dialéctico impone restricciones adicionales para que el intercambio de argumentos pueda ser 
considerado racionalmente \textit{aceptable}. Puede ser probado que las líneas argumentativas aceptables 
son finitas. Un tratamiento exhaustivo sobre restricciones de aceptabilidad pueden ser encontradas en el
trabajo de Garcia y Simari. %REF
Dado un programa \DLP\ \PP\ y un argumento inicial \AoQo, el conjunto de todas las líneas argumentativas
aceptables comenzando en \AoQo\ da lugar a un análisis dialéctico completo para \AoQo\ (\ie, todos los
diálogos posibles sobre \AoQo\ entre proponente y oponente), formalizado mediante un \textit{árbol dialéctico}.

Los nodos en un árbol dialéctico $T_{\scriptsize \AoQo}$ pueden ser marcados como nodos \textit{derrotados} 
y \textit{no derrotados} (nodos D -\textit{defeated}- y nodos U -\textit{undefeated}-, respectivamente). 
Un árbol dialéctico será marcado como un árbol {\footnotesize AND-OR}: todas las hojas en 
$T_{\scriptsize \AoQo}$ serán marcadas como nodos U (dado que no poseen derrotadores), y cada nodo interno 
será marcado como nodo D si y sólo si tiene al menos un nodo U como hijo, y como nodo U en otro caso. 
Un argumento \AoQo\ es finalmente aceptado como válido (o \textit{garantizado}) con respecto a un programa 
\DLP\ \PP si y sólo si la raíz del árbol dialéctico asociado $T_{\scriptsize \AoQo}$ está etiquetado como \textit{nodo U}. %Superíndice

Dado un programa \DLP\ \PP, resolver una consulta \ArgQ\ con respecto a \PP\ implica determinar si \ArgQ\ está soportado por (al menos) un argumento garantizado. Diferentes actitudes doxásticas %?
pueden ser distinguidas de la siguiente manera:

\begin{enumerate}[(1)]
	\item \textit{Yes}: se cree \ArgQ\ si y sólo si hay al menos un argumento garantizado soportando
		\ArgQ\ en base a \PP.
	\item \textit{No}: se cree \lit{\no \ArgQ} si y sólo si hay al menos un argumento garantizado 
		soportando \lit{\no \ArgQ} en base a \PP.
	\item \textit{Undecided}: ni \ArgQ\ ni \lit{\no \ArgQ} están garantizados con respecto a \PP.
	\item \textit{Unknown}: \ArgQ\ no se encuentra en el lenguaje de \PP.
\end{enumerate}
